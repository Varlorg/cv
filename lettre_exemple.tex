%%%%%%%%%%%%%%%%%%%%%%%%%%%%%
% Auteur : Adrian Poiget    %
% Créé le : 05/06/2014      %
% Mise à jour : 05/06/2014  %
% Version 1.0               %
%%%%%%%%%%%%%%%%%%%%%%%%%%%%%
% Inspiré par celui de Nicolas Gressier Créé le : 27/06/2007  Version 1.0
%%%%%%%%%%%%%%%%%%%%%%%%%%%%%

\documentclass[a4paper,sans,12pt]{article}

\usepackage[utf8]{inputenc}
%\usepackage[utf8x]{inputenc}
\usepackage[frenchb]{babel}
\usepackage[T1]{fontenc}
\usepackage{lmodern}
\usepackage{marvosym}
\usepackage{lipsum}
%\renewcommand{\baselinestretch}{1.2}

%\renewcommand{\sfdefault}{pgg}
%\renewcommand{\sfdefault}{ptm} %Times font
%\renewcommand{\sfdefault}{phv} %Helvetica font
%\renewcommand{\sfdefault}{bch} %Charter font
\renewcommand{\ttdefault}{pcr}

\pagestyle{empty}
\usepackage[top=2cm, bottom=2cm, left=2cm, right=2cm]{geometry}
\setlength{\parindent}{0pt}
\addtolength{\parskip}{10pt}
\def\interligne{1}% interligne

\def\lieu{Ville}
\def\signature{signature}
\def\expediteur{Nom expéditeur}
\def\adresseExpediteur{Adresse expéditeur }
\def\mail{Mail}
\def\mobile{num de téléphone}
\def\fixe{num tel fixe}
\def\destinataire{Destinataire }
\def\societe{societe}
\def\adresseDestinataire{ Adresse destinataire}
\def\objet{Définir objet }
\def\formuleAppel{Formule d'appel}

\begin{document}
%\sffamily
%\hfill
%
\begin{minipage}[t]{.6\textwidth}
    %\raggedleft
    \raggedright
    {\bfseries \expediteur}\\[.35ex]
    \small\itshape
    \adresseExpediteur\\
    %\Telefon~\fixe\\
    \Mobilefone~\mobile\\
    \Letter~\mail
\end{minipage}\\[1em]
\hfill
\begin{minipage}[t]{1\textwidth}
    %\raggedright
    \raggedleft
    {\bfseries \destinataire}\\[.35ex]
    \small\itshape
    \adresseDestinataire\\
\end{minipage}\\[2em]
%
\begin{flushleft}
\textbf{Objet : }{\normalsize \objet }
\end{flushleft}
%
\hfill
%
\begin{minipage}[t]{1\textwidth}
    \raggedleft
    \lieu, le \today
\end{minipage}\\[3em]
%
\formuleAppel,\\[0.2em]
{\setlength{\baselineskip}{\interligne\baselineskip} % augmenter l'interligne
% fini par : \par} %%% <= terminer le paragraphe
    \lipsum[1-3]
\par} %%% <= terminer le paragraphe
\vspace*{0.4cm}
\raggedleft 
\signature
\end{document}
